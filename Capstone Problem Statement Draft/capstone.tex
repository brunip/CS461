\documentclass[onecolumn, draftclsnofoot,10pt, compsoc]{IEEEtran}
\usepackage{graphicx}
\usepackage{url}
\usepackage{setspace}

\usepackage{geometry}
\geometry{textheight=9.5in, textwidth=7in}

% 1. Fill in these details
\def \CapstoneTeamName{			Dream Team}
\def \CapstoneTeamNumber{		57}
\def \GroupMemberOne{			Parker Bruni}
\def \GroupMemberTwo{			Daniel Schroeder}
\def \GroupMemberThree{			Aubrey Thennell}
\def \CapstoneProjectName{		A Scalable Web Application Framework for Monitoring Energy Usage on Campus}
\def \CapstoneSponsorCompany{	Oregon State University Sustainability Office}
\def \CapstoneSponsorPerson{	Jack Woods}

% 2. Uncomment the appropriate line below so that the document type works
\def \DocType{		Problem Statement
				%Requirements Document
				%Technology Review
				%Design Document
				%Progress Report
				}
			
\newcommand{\NameSigPair}[1]{\par
\makebox[2.75in][r]{#1} \hfil 	\makebox[3.25in]{\makebox[2.25in]{\hrulefill} \hfill		\makebox[.75in]{\hrulefill}}
\par\vspace{-12pt} \textit{\tiny\noindent
\makebox[2.75in]{} \hfil		\makebox[3.25in]{\makebox[2.25in][r]{Signature} \hfill	\makebox[.75in][r]{Date}}}}
% 3. If the document is not to be signed, uncomment the RENEWcommand below
\renewcommand{\NameSigPair}[1]{#1}

%%%%%%%%%%%%%%%%%%%%%%%%%%%%%%%%%%%%%%%
\begin{document}
\begin{titlepage}
    \pagenumbering{gobble}
    \begin{singlespace}
    	%\includegraphics[height=4cm]{coe_v_spot1}
        \hfill 
        % 4. If you have a logo, use this includegraphics command to put it on the coversheet.
        %\includegraphics[height=4cm]{CompanyLogo}   
        \par\vspace{.2in}
        \centering
        \scshape{
            \huge CS Capstone \DocType \par
            {\large\today}\par
			{\large CS461 Fall 2017}\par
            \vspace{.5in}
            \textbf{\Huge\CapstoneProjectName}\par
            \vfill
            {\large Prepared for}\par
            \Huge \CapstoneSponsorCompany\par
            \vspace{5pt}
            {\Large\NameSigPair{\CapstoneSponsorPerson}\par}
            {\large Prepared by }\par
            Group\CapstoneTeamNumber\par
            % 5. comment out the line below this one if you do not wish to name your team
            \CapstoneTeamName\par 
            \vspace{5pt}
            {\Large
                \NameSigPair{\GroupMemberOne}\par
                %\NameSigPair{\GroupMemberTwo}\par
                %\NameSigPair{\GroupMemberThree}\par
            }
            \vspace{20pt}
        }
        \begin{abstract}
        % 6. Fill in your abstract    
        	This document outlines the problem statement for the development of a scalable web application intended to monitor
			energy usage of Oregon State campus buildings. Oregon State has up to 5 energy monitoring sensors installed in 28 different buildings 
			and the application will gather the data from those sensors and display it in a formal and readable way.  This application will serve as a replacement
			to the current system, which is becoming too costly to continue as more buildings are added to the campus.
			The energy usage data will be presented in a way that will allow analytical decisions to be made with regards to 
			future building projects as well as provide live updates of the data in meaningful formats.
        \end{abstract}     
    \end{singlespace}
\end{titlepage}
\newpage
\pagenumbering{arabic}
\tableofcontents
% 7. uncomment this (if applicable). Consider adding a page break.
%\listoffigures
%\listoftables
\clearpage

% 8. now you write!
\section{Definition and Description of Problem}

	As the population of the world grows exponentially, the demand for energy grows with it. Unfortunately, a massive source of energy for humans comes from
	the burning of carbon based fuels. This burning releases green house gasses into the earth's atmosphere, which results in the heating of our planet and
	many consequences that come as a result. To combat this, communities have decided that it is time to get our energy from more sustainable sources, as well
	as use it in more sustainable ways. In order to be more sustainable and concious of our energy consumption, we need to implement modern tools to monitor
	energy usage data and use that data to make informed decisions on future infrastructure projects. This is necessary to reduce our carbon footprint, reduce costs, and move
	society to a more sustainable (and eventually fully sustainable) future in regards to energy consumption. 
	
	Oregon State University is a university that values energy efficiency and sustainability and aims to make more sustainable energy decisions when starting new infrastructure projects.
	In this era of technology it is necessary to utilize the tools that are available to society to design and implement systems and solutions that will increase the energy efficiency of Oregon State's 
	campus as a whole, to reduce costs as well as support a more sustainable environment for society. As such, energy meters have been installed in buildings 
	to monitor and record energy data so that it may be analysed by members of Oregon State University's Sustainability Office. The data that is gathered 
	from these systems then be used to make educated decisions about the current energy usage systems in place as well as provide a foundation of knowledge 
	to reference when planning future infrastructure projects. In order for the data to be effectively monitored by those who may make these decisions, a web 
	application has been installed by the company Lucid, but as OSU scales up the automated energy reporting infrastructure, the contract that is currently in place
	with Lucid will become too expensive to continue. In order for this data analysis to continue, the Oregon State Sustainability office has recognized an 
	opportunity for OSU engineering students to create and plan a project that can take the place of Lucid's system to use as their senior capstone project. 
	This is an effective solution to the problem as it could potentially eliminate the cost of monitoring the energy usage of Oregon State University buildings
	while simultaneously providing students with a fun and useful project to develop as their senior capstone project.

\section{Proposed Solution}
	In order for our group to effectively replace the currently existing costly application developed by Lucid we must create an organized plan and procedure
	for developing an application that may serve in its place. The current application will need to be replaced by a clean, attractive, and intuitive user interface to access the 
	data as well as to view the data in meaningful ways such as average energy usage over various time-frames, graphs that demonstrate trend lines based on
	desired trends, energy usage per building, energy usage per meter, average energy usage by building per season, and more. The application will also need to 
	have a solid back-end framework in order to accurately utilize the hard data gathered from the energy meters of each building. Each member of the group
	will focus on a different part of the application in order to finish the application in a timely manner. Communication between team members 
	as well as to the client will be absolutely necessary to ensure that everyone is producing in a way that meets the requirements of the project while 
	still managing time appropriately. The application will need to be accessible from the web by members of the Oregon State Sustainability Office who 
	will have the ability to create personal accounts, as well as other members of Oregon State University staff or students that should need access to the
	energy usage information. Within the personal accounts, a user may be able to customize their own personal dashboard to their preference so that they
	may have a more efficient viewing of the data as well as to reduce cluttered information that is not relevant to them. By using this application, members of 
	the Oregon State Sustainability Office will be able to easily interpret the data gathered by the energy meters within the building and make decisions based on
	those interpretations that will make planning future infrastructure projects much easier, cost effective, and sustainable. 
	
\section{Performance Metrics}
	For this project to be considered a complete and successful project, there are various criterion that need to be met. The application must first and 
	foremost be scalable, as dictated in the project title "A Scalable Web Application Framework for Monitoring Energy Usage on Campus". This implies that the
	application has the ability for more buildings and more energy meter nodes to be easily added to the framework. If adding buildings or meter nodes to the
	system breaks the application, or if it takes great effort to add them, the application is not complete. Another trait that the website should have is 
	a intuitive and clean looking UI. Not only is it important to process and view the data, it is necessary to present it in a way that is visually appealing
	to better the experience of the user. Along with a clear UI, the data that is displayed in graphs or trend lines must be accurate, relevant, and interpretable.
	A major purpose of this application is to make judgement decisions on projects based off of the data, if the data is incorrect, the projects that may be created
	and planned based off of this data may contain costly errors if put into production. The data should be able to be manipulated to the users liking. For example, 
	if a user wants to view the average trend line on a graph of the energy consumption of Kelley Engineering Center over a 12-month period in 2014, the user should
	be able to quickly navigate the application and manipulate options to have that data presented to them. As for the back end of this web application, the data 
	from the energy monitoring meters in the buildings must be accurately and quickly updated as they gather that data. Users of this site should also be able to 
	create personal accounts that give them UI dashboard customizability so they may quickly view data that is relevant to them. And finally, this application should
	effectively serve as a replacement for the current application that is in place. The savings from replacing the current system will be massively beneficial to 
	the Oregon State Office of Sustainability and for the Oregon State University as a whole.
	
\end{document}





